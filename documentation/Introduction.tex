\section{Introduction}
\label{SetUp}
The uncertainty of Cherenkov photons time from dispersion is estimated as $R\frac{\delta v_C}{v_C^2}$, in which $R$, $\delta v_C$, and $v_C$ are the radius of the detector, the uncertainty of velocity, and center value of the velocity of Cherenkov light. The spectrum of Cherenkov light spread from \SI{300}{nm} to \SI{700}{nm} \cite{Luo:2022xrd}. Considering the uncertainty from dispersion is about \SI{1.3}{ns} for \SI{10}{m} scale detectors\footnote{The values are estimated as \SI{10}{m}, \SI{5}{mm/ns} and \SI{192}{mm/ns} from \cite{Luo:2022xrd}.}, the short time scale of Cherenkov light requires the deviation of transit time to be at the scale of \SI{1}{ns}.
\subsection{Experimental setup}
\label{sec:setup}
% Facility

The light source is a picosecond laser flashing system (PiL040XSM) from Advanced Laser Systems (A.L.S) \cite{NTKLaser}, which can produce short light pulses with a wavelength of \SI{405}{nm} and a width of \SI{34}{ps}. Besides, a period trigger signal output from the laser is connected to the digitizer, which serves as the trigger input. An attenuator is used to adjust the laser intensity. The light is attenuated by a laser attenuator with an adjustable range of $[0.5,30]$\,DB \cite{Attenuator}.

CAEN V1751 10-bit digitizer with 1GHz sample rates is used for data taking \cite{CAENV1751}, which is controlled by a custom-made data acquisition (DAQ) software\footnote{Github repo: \url{https://github.com/greatofdream/CAENReadout}} based on CAEN library. The dynamic range of V1751 is \SI{1}{V} and \SI{1}{LSB} (Least Significant Byte) is about \SI{0.978}{mV}, which is defined as \SI{1}{ADC}. To handle the anomalous baseline larger than 0V, the offset of the digitizer is set to about \SI{0.183}{V} and the dynamic range is from \SI{-0.817}{V} to \SI{0.183}{V}. Wiener EDS 30330p high voltage (HV) module \cite{WIENERHV} supplies a positive voltage for each PMT. HV and laser are controlled by a computer using SNMP \cite{SNMP} and VISA \cite{VISA} protocol.

A black plastic box split into 4 grids with clapboards acts as a darkroom. A fiber splitter with the standard G.657.A1 distributes attenuated light into 4 channels \cite{Splitter}. Each grid holds a PMT and is equipped with an end of splitter channel, which faces directly to the top of the PMT. The end of the splitter channel is covered with a \SI{4}{cm\tothe{2}} square diffuser plate to diffuse the laser light into a small area. Potted PMTs are connected from an integrated HV-signal decoupler in the PMT base to the HV module and digitizer with the HV lines and LEMO lines. Optical fibers and LEMO lines go through a hole with a diameter of \SI{0.5}{cm} on the box. Double-way connectors are used to connect HV lines through the box. Besides, CR365 PMT \cite{BJBS} from HAMAMATSU is used as a reference PMT with \SI{2.4}{ns} of TTS and 25\%@\SI{420}{nm} of quantum efficiency for design.

\subsection{Testing procedures}

The testing procedures are indicated in Fig.~\ref{fig:testingprocedure}. All PMTs were measured under the HV provided by the vendor (NNVT), which has been calibrated at the gain of $1\times10^7$ by vendor. 
The PMT-box map is stored as meta data which is mainly used for PDE and testing system calibration. To stabilize and reduce the influence of the dark noise, all PMTs stay in the darkroom with HV on for at least 12 hours before running of DAQ software.

Two tests modes are designed for different purposes. One is dark noise tests, in which PMTs work without laser and the digitizer is triggered periodly at 1kHz mainly for DCR calculation; the other is laser tests, in which the digitizer samples data with laser on and is triggered by the laser during the experiment. The procedures are executed by the software automaticaly.

The waveforms are stored in ROOT files and analyzed with the custom-made software\footnote{Github repo: \url{https://github.com/greatofdream/pmtTest}}.

\subsection{Setup of dark noise test}
The sample window size $T_{\mathrm{wave}}$ is set as \SI{600}{ns} in the noise stage. For \SI{20}{kHz} dark noise rate, the number of dark noise pulses in each waveform obeys Poisson distribution $\mathrm{\pi}(\nu=0.012)$, which means the probability of 2 or more pulses is about 0.006 times of the probability of 1 pulse\footnote{The probability of n dark noise pulses is $P(n)=e^{-\nu}\frac{\nu^n}{n!}$, $P(n=1)=0.012$ and $P(n>1)=7.14\times10^{-5}$.} and single dark noise pulse events dominate the dataset. Considering the pulse width is far shorter than $T_{\mathrm{wave}}$, the probability of pulses overlapping with each other is less than 0.006 and the overlapping cases are omitted.

The peak time $t_p$ is the position of minimum in each waveform. Because the maximum pulse is selected in each waveform, the expected charge distribution is the distribution of $\max_n^N(C_n)$, in which $C_n$ is the charge of nth PE and $N$ is the number of pulses in a waveform. The charge bias caused by multi-pulse is omitted because of the small ratio of multi-pulse cases. Due to the offset mentioned in sec.~\ref{sec:setup}, the value of the baseline is not zero. The baseline is calculated from an interval of the time window $[-t_s,-10]$\,ns ($t_s\leq200$) relative to pulse peak $t_p$. If the peak is close to the start of the waveform and therefore $t_s < 110$, another interval $[t_p+100,t_p+200]$\,ns is appended to the total interval to ensure the length of the interval is enough. Average $\mu_{\mathrm{b0}}$ and standard deviation $\sigma_{\mathrm{b0}}$ of amplitudes in the interval are calculated. A baseline threshold filter $[\mu_{\mathrm{b0}}-\max(\min(5\sigma_{\mathrm{b0}},3),1)]$ is used to remove potential signals and reserves the baseline when $\sigma_{\mathrm{b0}}$ is small. The selected areas are cut off \SI{10}{ns} at both ends to remove the rising edge and falling edge of potential pulses. The baseline $\mu_b$ and fluctuation $\sigma_b$ are estimated as average and the standard deviation of the rest waveform.

\subsection{Setup of laser test}
To yield single photoelectron (SPE) events, the laser intensity was adjusted to a level only about one out of 20 triggers lead to a signal. The window size $T_{\mathrm{wave}}$ is set to \SI{10400}{ns} and the rising edge of trigger waveform is at about \SI{200}{ns}, which reserves a time interval for pre-pulse analysis. The trigger from the laser system is a step wave, of which the vertical center of rising edge is linearly interpolated to get the trigger time $t_{\mathrm{trig}}$.

The analysis window size $T_{\mathrm{wave}}$ is set as \SI{600}{ns} in the preanalysis. For \SI{20}{kHz} dark noise rate, the number of dark noise pulses in each waveform obeys Poisson distribution $\mathrm{\pi}(\nu=0.012)$, which means the probability of 2 or more pulses is about 0.006 times of the probability of 1 pulse\footnote{The probability of n dark noise pulses is $P(n)=e^{-\nu}\frac{\nu^n}{n!}$, $P(n=1)=0.012$ and $P(n>1)=7.14\times10^{-5}$.} and single dark noise pulse events dominate the dataset. Considering the pulse width is far shorter than $T_{\mathrm{wave}}$, the probability of pulses overlapping with each other is less than 0.006 and the overlapping cases are omitted.

The peak time $t_p$ is the position of minimum in each waveform as shown in Fig.~\ref{fig:triggertime}. Because the maximum pulse is selected in each waveform, the expected charge distribution is the distribution of $\max_n^N(C_n)$, in which $C_n$ is the charge of nth PE and $N$ is the number of pulses in a waveform. The charge bias caused by multi-pulse is omitted because of the small ratio of multi-pulse cases. Due to the offset mentioned in sec.~\ref{sec:setup}, the value of the baseline is not zero. The baseline is calculated from an interval of the time window $[-t_s,-10]$\,ns ($t_s\leq200$) relative to pulse peak $t_p$. If the peak is close to the start of the waveform and therefore $t_s < 110$, another interval $[t_p+100,t_p+200]$\,ns is appended to the total interval to ensure the length of the interval is enough. Average $\mu_{\mathrm{b0}}$ and standard deviation $\sigma_{\mathrm{b0}}$ of amplitudes in the interval are calculated. A baseline threshold filter $[\mu_{\mathrm{b0}}-\max(\min(5\sigma_{\mathrm{b0}},3),1)]$ is used to remove potential signals and reserves the baseline when $\sigma_{\mathrm{b0}}$ is small as shown in Fig.~\ref{fig:triggertime}. The selected areas are cut off \SI{10}{ns} at both ends to remove the rising edge and falling edge of potential pulses. The baseline $\mu_b$ and fluctuation $\sigma_b$ are estimated as average and the standard deviation of the rest waveform.

The triggered pulse is mainly centered in the time interval between $[t_{\mathrm{trig}}, 600]$\,ns dependent on the length of the cable. The maximum peak is found in the window of $[t_{\mathrm{trig}}, 600]$\,ns to roughly extract the peak position $t_p$. A gaussian function G$(\mu_{t0},\sigma_{t0})$ is unbinned fitted to the distribution of peak location of pulses whose peak is larger than \SI{5}{ADC} for each PMT.

All the characterizations include $t_p$ are calculated with the new time cut $[\mu_{t0}-5\sigma_{t0}, \mu_{t0}+5\sigma_{t0}]$, which reduces the impact of dark noise.

% \begin{figure}[!htbp]
%     \centering
%     \includegraphics[width=0.4\textwidth]{figures/method/triggerpeakpos.pdf}
%     \caption{Peak location distribution of an example MCP-PMT. A gaussian function is fit to the distribution to acquire an optimized time cut.}%PM
%     \label{fig:peaklocation}
% \end{figure}
 Optical fibers and signal lines go through a hole with a diameter of \SI{0.5}{cm} on the box. Double-way connectors are used to connect HV lines through the box.
 The test procedures are indicated in Fig.~\ref{fig:testingprocedure}. All PMTs were measured under the HV provided by the vendor (NNVT), which has been calibrated at the gain of $1\times10^7$. 
The PMT-box map is stored as meta data which is mainly used for PDE and testing system calibration. To stabilize and reduce the influence of the dark noise, all PMTs stay in the darkroom with HV on for at least 12 hours before running of DAQ software. Testing contains two stages: the first stage is the noise stage, in which PMTs work without laser and the digitizer is triggered periodly at 1kHz mainly for DCR calculation; the second stage is laser stage, in which 
The digitizer samples data with laser on and is triggered by the laser during the experiment. The waveforms are stored in ROOT files and analyzed with the custom-made software\footnote{Github repo: \url{https://github.com/greatofdream/pmtTest}}.